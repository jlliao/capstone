\documentclass[11pt, oneside, a4paper]{article}


\input{preamble.tex}
%---header/style/enumeration-------
\pagestyle{fancy}
\lhead{Capstone}
\chead{MCS}
\rhead{2019-2020 Semester 1}
\author{
    Student: Jianglong LIAO\\
    Supervisor: Professor Bruno Bodin
    }
%----------------------------------

%-----------MY INFORMATION---------
%\title{{\fontfamily{qbk}\selectfont
\title{\textsc{Capstone Proposal}}
\date{\vspace{-5ex}} 
%----------------------------------
%unnumbered sections in TOC
\setcounter{secnumdepth}{0}

%-----This is where the GOOD STUFF begins---
\begin{document}

\maketitle

\thispagestyle{fancy}


\section{Title}
Real-time filtering for benchmarking multi-sensory SLAM systems

\section{Subject Areas}
Software engineering, computer vision, robotics, embedded systems, benchmarking.

\section{Challenges}
\subsection{Theoretical knowledge}
Understanding and integrating the filtering algorithm requires a strong basis on linear algebra and image processing methods.

The student has taken advanced courses such as machine learning and computer-assisted image processing and familiarized himself with topics such as simultaneous localization and mapping, embedded systems, and advanced algorithms via previous internships, independent reading, and self-studying.

\subsection{Implementation}
Implementing the real-time filtering in C++ in the large, on-going, open-source project SLAMBench requires a thorough understanding of software architecture, coordination ability with the global research team and versioning control.

The student has done numerous open source projects in Linux environment, especially in the area of performance benchmarking, and is currently taking the courses such as software engineering and C. Although this will be his first attempt to develop a module in C++, previous experiences in other object-oriented languages (Python, R, Java, Javascript) should assist with the process.

\section{Scope}
Benchmarking a SLAM system for each specific type of SLAM algorithm requires a combination of deep understanding of computer vision and special coding adjustment for various filters. Creating and integrating a generic filtering module into a SLAM benchmarking system allows SLAM researchers to test their SLAM systems with little specific knowledge of certain filtering techniques and start benchmarking with minimum configuration.

The first goal of this project is to investigate the current SLAMBench package and consider restructuring the architecture with minimum impact on the end users. Then implement three fundamental filters (random drop, blur, resize) to test their compatibility with the SLAMBench program. The second goal is to develop a more versatile filtering system using machine learning / advanced algorithms for benchmarking purposes. 

\section{Expectations associated with grade achievement}

A detailed description of the restructuring process of the SLAMBench program demonstrates adequate understanding of the SLAM benchmarking, and software engineering as it is essential for integrating the filtering module.

\noindent A filtering module compatible with SLAMBench program 

\section{Semester 1 plan with time allocation}

Estimated consultation time is one hour per week. Consultation consists of communication via git repository as well as one-on-one meetings.

\begin{center}
    \begin{tabular}{|l|l|l|} 
        \hline
        Time & Task & Deliverable\\
        \hline
        week 1 & Draft project proposal. & \\ 
        \hline
        week 2 & Finalize proposal with supervisor.& \\ 
        \hline
        week 3 & Learn about the architecture of SLAMBench program.&\\
        \hline
        week 4 & Start restructuring the architecture of SLAMBench program.&\\
        \hline
        week 5 & Read papers on the SLAM filtering and discuss the restructuring of program.&\\
        \hline 
        16 Sep 5pm & & Proposal\\
        \hline
        week 6 & Finish the first filter on SLAMBench and background information for paper.&\\
        \hline
        week 7 & Test the first filter, and evaluate the result.&\\
        \hline
        week 8 & Write a report on the benchmarking result of the first filter.&\\
        \hline
        week 9 & Think about the second filter and start drafting report 1.&\\
        \hline
        18 Oct 5pm & & Draft Report 1\\
        \hline
        week 10 & Implement the second filtering in SLAMBench.&\\
        \hline
        week 11 & Organize the filtering and run testing in SLAMBench.& Presentation\\
        \hline
        week 12 & Write a documentation for the 2 filters.& Presentation\\
        \hline
        week 13 & Write a documentation for the filtering.&\\
        \hline
        15 Nov 5pm & & Final Report 1\\
        \hline
    \end{tabular}
\end{center}

\end{document}